\documentclass{article}
\usepackage[utf8]{inputenc}
\usepackage{amsmath}
\usepackage{graphicx}
\usepackage{geometry}
\graphicspath{{images/} 
\geometry{legalpaper, lmargin=0.7in, bmargin=1in}}

\begin{document}
%%%%%%%%%%%%%%
%page  titre en caractères plus large
%%%%%%%%%%%%%%
\begin{titlepage}   
	\large{
		\begin{center}
			UNIVERSITÉ DE SHERBROOKE\\Faculté de génie\\
			Département de génie électrique et génie informatique\\
			\vspace{3cm}
			{\LARGE\textbf{éléments de statique et dynamique}}\\
			\vspace{2cm}
			\LARGE{Rapport APP1}\\
			\vspace{2cm}
			Présenté à\\l'équipe professorale de la session S4\\
			\vspace{2cm}
			Produit par\\Axel Bosco, Philippe Garneau, Philippe Spino\\
			\vspace{1cm}
			\vfill{7 mai 2017 - Sherbrooke}
		\end{center}
	}
\end{titlepage}
\newpage
%%%%%%%%%%%%%%
%Table des matières
%%%%%%%%%%%%%%
\tableofcontents

\newpage
\section{Introduction}
Dans le cadre de l’implémentation d’un système de commande du bras mécanique de l’entreprise CRM, il faut analyser le mouvement d’un point A sur le plan 2D de celui-ci. Le point A, situé à l’extrémité des bras du robot, bouge selon le bras BA attaché au moteur MB et le bras BA bouge selon le bras OB avec le moteur MO. Notre mandat est de déterminer les forces et les couples nécessaires pour maintenir le robot en équilibre ou de le bougé selon des directives spécifiques. Pour la résolution de la problématique, l’équipe a divisé l’ensemble en plusieurs étapes. La première étape fût de regarder la cinématique du système de manière générale, ensuite dans des cas avec des restrictions sur les mouvements possibles du point A dans le plan 2D. En deuxième partie, l’analyse est centrée sur la statique et la dynamique du système. 

\section{Cinématique}
Dans l'analyse de la cinématique, il y avait trois cas à analyser. Il faut déterminer la relation du mouvement du Point A en reliant les mouvements angulaires des bras OB et BA au mouvement linéaire du point A dans tous les cas. Précisément, il faut déterminer les vecteurs de positions, de vitesses et d’accélération linéaire du point A en fonction des longueurs des bras, soit L1 et L2, des angles $\phi$ et $\theta$ et de leur vitesse et accélération angulaires respectives.
Les calcules présenté dans la section suivant explique les démarches mathématiques utilisé pour la résolution des cas.

\subsection{Équations générales}
équation générale: vecteur position

\begin{equation}
\overrightarrow{OA} = 
    l_1\Bigg[\begin{array}{cc}
    cos(\theta) \\
    sin(\theta) \\
    0
    \end{array}\Bigg]
    +
    l_2\Bigg[\begin{array}{cc}
    cos(\phi) \\
    sin(\phi) \\
    0
    \end{array}\Bigg]
\end{equation}

\noindent équation générale: vecteur vitesse

\begin{equation}
\overrightarrow{V_{OA}} = 
    l_1\Bigg[\begin{array}{cc}
    -sin(\theta)\theta' \\
    cos(\theta)\theta' \\
    0
    \end{array}\Bigg]
    +
    l_2\Bigg[\begin{array}{cc}
    -sin(\phi)\phi' \\
    cos(\phi)\phi' \\
    0
    \end{array}\Bigg]
\end{equation}

\noindent equation generale: vecteur acceleration

\begin{equation}
\overrightarrow{\alpha_{OA}} =
	l_1\Bigg[\begin{array}{cc}
	-cos(\theta)(\theta')^2-sin(\theta)\theta'' \\
	-sin(\theta)(\theta')^2+cos(\theta)\theta'' \\
	0
	\end{array}\Bigg]
	+
	l_2\Bigg[\begin{array}{cc}
	-cos(\phi)(\phi')^2-sin(\phi)\phi'' \\
	-sin(\phi)(\phi')^2+cos(\phi)\phi'' \\
	0
	\end{array}\Bigg]
\end{equation}

\subsection{Mouvement horizontal}
\subsubsection{Relation entre $\theta$ et $\phi$ lorsque $\phi$ est negatif}
Trouver $sin(\phi)$:

\begin{equation}
l_1 = l_2
\end{equation}

\begin{equation}
\overrightarrow{Y_A} = l_1sin(\theta)+l_1sin(\phi)
\end{equation}

\begin{equation}
0 = l_1sin(\theta)+l_1sin(\phi)
\end{equation}

\begin{equation}
sin(\phi) = -sin(\theta)
\end{equation}

\noindent Trouver $cos(\phi)$ a partir de $sin(\phi)$:

\begin{equation}
cos^2(\phi)+sin^2(\phi) = 1
\end{equation}

\begin{equation}
cos(\phi) = \sqrt{1-sin^2(\theta)}
\end{equation}

\subsubsection{Équations cinématiques}
Position:

\begin{equation}
l_1 = l_2
\end{equation}

\begin{equation}
\overrightarrow{X_A} = l_1cos(\theta)+l_1\sqrt{1-sin^2(\theta)}
\end{equation}

\begin{equation}
\overrightarrow{X_A} = l_1cos(\theta)+l_1\sqrt{cos^2(\theta)}
\end{equation}

\begin{equation}
\overrightarrow{X_A} = 2l_1cos(\theta)
\end{equation}

\noindent Vitesse:

\begin{equation}
\overrightarrow{V_{Ax}} = \frac{d(2l_1cos(\theta))}{dt}
\end{equation}

\begin{equation}
\overrightarrow{V_{Ax}} = -2l_1sin(\theta)\theta'
\end{equation}

\begin{equation}
\theta' = \omega_{OB}
\end{equation}

\begin{equation}
\overrightarrow{V_{Ax}} = -2l_1sin(\theta)\omega_{OB}
\end{equation}

\noindent Acceleration:

\begin{equation}
\overrightarrow{\gamma_{Ax}} = \frac{d(-2l_1sin(\theta)\theta')}{dt}
\end{equation}

\begin{equation}
\overrightarrow{\gamma_{Ax}} = -2l_1cos(\theta)(\theta')^2-2l_1sin(\theta)\theta''
\end{equation}

\begin{equation}
\theta' = \omega_{OB}
\end{equation}

\begin{equation}
\theta'' = \alpha_{OB}
\end{equation}

\begin{equation}
\overrightarrow{\gamma_{Ax}} = -2l_1cos(\theta)(\omega_{OB})^2-2l_1sin(\theta)\alpha_{OB}
\end{equation}

\subsubsection{Courbes du mouvement horizontal}
Position:
\newline
\noindent \includegraphics[width=\textwidth]{hori_pos}
Avec les contraintes, la position de $A_y$ doit être fixée à la hauteur du point O en tout temps. Ce qui oblige un mouvement horizontal seulement. Selon le graphique, la position initiale en $A_x$ est à distance de $L1 + L2$. Plus $\theta$ est grand, plus le déplacement horizontal est rapide. La position finale atteinte à $\pi/3$ est à distance de L1 du point O.
\newline
\newline
\noindent Vitesse:
\newline
\noindent \includegraphics[width=\textwidth]{hori_vit}
Initialement au repos, la vitesse varie et accélère en fonction de $\theta$. Cependant, l'accélération est de moins en moins grande au fil du temps. Puisque le déplacement est négatif, la vitesse est négative au fil du déplacement. À la configuration finale du robot, la vitesse de son déplacement horizontal est environ $-11 m/s$.
\newline
\newline
\noindent Acceleration:
\newline
\noindent \includegraphics[width=\textwidth]{hori_accel}
À l'instant où le robot a commencé son déplacement horizontal, l'accélération est à son maximum. À mesure que $\theta$ grandit, l'accélération diminue. Puisque les angles sont de plus en plus grand, la distance séparant le point O au point A est de moins en moins grande. L'angle doit être plus grand pour effectuer plus de déplacement.

\subsubsection{Configuration initiale et finale}
Initiale:
\newline
\centerline{\noindent \includegraphics[scale=0.75]{IH}}
\newline
\newline
\noindent Finale:
\newline
\centerline{\noindent \includegraphics[scale=0.75]{FH}}
\newline

\subsection{Mouvement vertical}
\subsubsection{Relation entre $\theta$ et $\phi$ lorsque $\phi$ est negatif}
Trouver $cos(\phi)$:

\begin{equation}
l_1 = l_2
\end{equation}

\begin{equation}
\overrightarrow{X_A} = l_1cos(\theta)+l_1cos(\phi)
\end{equation}

\begin{equation}
l_1 = l_1cos(\theta)+l_1cos(\phi)
\end{equation}

\begin{equation}
cos(\phi) = 1-cos(\theta)
\end{equation}

\noindent Trouver $sin(\phi)$ a partir de $cos(\phi)$:

\begin{equation}
cos^2(\phi)+sin^2(\phi) = 1
\end{equation}

\begin{equation}
sin^2(\phi) = 1-cos^2(\phi)
\end{equation}

\begin{equation}
sin^2(\phi) = -cos^2(\theta)+2cos(\theta)
\end{equation}

\begin{equation}
\pm sin(\phi) = \sqrt{-cos^2(\theta)+2cos(\theta)}
\end{equation}

\noindent Nous considerons que $\phi$ est negatif, donc:

\begin{equation}
sin(\phi) = -\sqrt{-cos^2(\theta)+2cos(\theta)}
\end{equation}

\subsubsection{Équations cinématiques}
Position:

\begin{equation}
l_1 = l_2
\end{equation}

\begin{equation}
\overrightarrow{Y_A} = l_1sin(\theta)-l_1\sqrt{-cos^2(\theta)+2cos(\theta)}
\end{equation}

\noindent Vitesse:

\begin{equation}
\overrightarrow{V_{Ay}} = \frac{d(l_1sin(\theta)-l_1\sqrt{-cos^2(\theta)+2cos(\theta)})}{dt}
\end{equation}

\begin{equation}
\overrightarrow{V_{Ay}} = l_1cos(\theta)\theta'-\\
\frac{l_1(-cos^2(\theta)+2cos(\theta))'}{2\sqrt{-cos^2(\theta)+2cos(\theta)}}
\end{equation}

\begin{equation}
\overrightarrow{V_{Ay}} = l_1cos(\theta)\theta'-\\
\frac{l_1sin(\theta)(cos(\theta)-1)\theta'}{\sqrt{-cos^2(\theta)+2cos(\theta)}}
\end{equation}

\begin{equation}
\theta' = \omega_{OB}
\end{equation}

\begin{equation}
\overrightarrow{V_{Ay}} = l_1cos(\theta)\omega_{OB}-\\
\frac{l_1sin(\theta)(cos(\theta)-1)\omega_{OB}}{\sqrt{-cos^2(\theta)+2cos(\theta)}}
\end{equation}

\subsubsection{Courbes du mouvement vertical}
Position:
\newline
\noindent \includegraphics[width=\textwidth]{vert_pos}
Selon les contraintes, la configuration initiale exige que $A_x$ est toujours à distance de L1 du point O. Puisque la relation entre $\theta$ et $\phi$ est lorsque $\phi$ est négatif, la position $A_y$ est à L2 de distance du point O pour la mouvement vertical. Le déplacement est linéaire et permet de trouver la configuration finale du robot à $\theta/3$, en respectant la contrainte que $A_x$ est toujours à la même distance de L1.
\newline
\newline
\noindent Vitesse:
\newline
\noindent \includegraphics[width=\textwidth]{vert_vit}
Puisqu'il existe la contrainte sur $A_x$, la vitesse varie dépendamment des angles $\theta$ et $\phi$ et doit . À environ $\theta = 0,7$, la courbe de vitesse change et accélère, alors que précédemment, elle décélérait.

\subsubsection{Configuration initiale et finale}
Initiale:
\newline
\centerline{\noindent \includegraphics[scale=0.75]{IV}}
\newline
\newline
\noindent Finale:
\newline
\centerline{\noindent \includegraphics[scale=0.75]{FV}}
\newline

\subsection{Relation entre $\theta$ et $\phi$ et les commandes de $M_O$ et $M_B$}
Lorsque le moteur $M_B$ s'active, il affecte seulement la valeur de $\phi$. Par contre, lorsque le moteur $M_O$ s'active, il vient bien evidemment modifier la valeur de $\theta$, mais aussi la valeur de $\phi$, car les tiges OB et BA peuvent etre consideree comme une seule tige OA lorsque seulement le moteur $M_O$ est en marche.

\section{Statique et Dynamique}
Pour l'analyse du système dans le domaine du statique, on considère le cas ou le robot porte un objet $O_A$ au point A. Pour simplifier l'analyse, les tiges, représenté par les vecteurs $\overrightarrow{OB}$ et $\overrightarrow{BA}$, sont approximés par des tiges minces et uniformes, les moteurs $M_O$, $M_B$ et $O_A$ sont approximés par des sphères de dimensions négligeables par rapport a OB et BA. On considère aussi que la force FB et le couple CB sont exercés sur l’extrémité B de la tige BA. FB est appliquée par OB alors que CB est appliqué par MB.

\subsection{Statique}
Dans le domaine statique, on fait l'étude du système a l'équilibre. C'est a dire lorsque: 
\begin{equation}
\sum \overrightarrow{F_{ext}} = 0
\end{equation}

\begin{center} 
et quand:
\end{center}

\begin{equation}
\sum \overrightarrow{M_B} = 0
\end{equation}
\newline
\newline
DCL du bras BA:
\newline
\noindent \includegraphics[width=\textwidth, scale = 0.01]{DCL_BA}

\noindent En appliquant les forces dans le Diagramme des Corps Libres, l'équation suivant est obtenu:

\begin{equation}
\sum \overrightarrow{F_{ext}} = -m_{BA}.\overrightarrow{g} -m_A.\overrightarrow{g} + \overrightarrow{F_B}
\end{equation}

\begin{equation}
\sum F_x = -m_{BA}.g_x -m_A.g_x + F_{B_x} = 0
\end{equation}

\begin{equation}
\sum F_y = -m_{BA}.g_y -m_A.g_y + F_{B_y} = 0
\end{equation}

\noindent Dans l'équation 41, il n'est pas nécessaire de calculer la valeur de $F_{B_x}$ car celle-ci vaut 0. Cela veut dire qu'il ne reste que $F_{B_y}$ comme force.

\begin{equation}
\ F_{B_y} = m_A.g + m_{BA}.g
\end{equation}

\noindent Il reste maintenant qu'a trouver la somme des moments de forces $\sum \overrightarrow{M_B}$.

\begin{equation}
\sum \overrightarrow{M_B} = -lm_Acos(\phi) - \frac{l}{2}m_{BA}gcos(\phi) + C_B = 0
\end{equation}

\noindent ce qui nous donne le couple $C_B$ suivant: 

\begin{equation}
\ C_B = lm_Acos(\phi)+ \frac{l}{2}m_{BA}gcos(\phi)
\end{equation}

\subsection{Dynamique}
Dans l'analyse du domaine dynamique du système, il faut faire l'étude du mouvement engendré par le moteur $M_O$. En regardant le Diagramme Cinétique du bras BA, on peut déterminer $F_B$ et $C_B$ en fonction de l'angle $\phi$, de $l$ dans le cas ou BA tourne avec une accélération angulaire constante $\alpha_{BA}$ pendant que OB est immobile.
\newline
\newline
\noindent DC du bras BA:
\newline
\noindent \includegraphics[width=\textwidth, scale = 0.01]{DC_BA}

\noindent En appliquant les forces dans le Diagramme Cinétique, l'équation suivante est obtenue:
\begin{equation}
\sum \overrightarrow{F_{ext}} = m_{BA}.\gamma_{G_{BA}} + m_A.\gamma_{G_A} + \overrightarrow{F_B}
\end{equation}

\begin{equation}
    \Bigg[\begin{array}{cc}
    F_{B_x} \\
    F_{B_y} \\
    0
    \end{array}\Bigg]
    =
    m_{BA}\Bigg[\begin{array}{cc}
    -\omega^2_{BA}\frac{l}{2} \\
    \alpha_{BA}\frac{l}{2} \\
    0
    \end{array}\Bigg]
    +
    m_A\Bigg[\begin{array}{cc}
    -\omega^2_{BA}l \\
    \alpha_{BA}l \\
    0
    \end{array}\Bigg]
\end{equation}

\noindent Ensuite, un projection est fait:

\begin{equation}
\ F_{B_x} = \frac{-m_{BA}\omega_{BA}l}{2} - m_A\omega_{BA}l 
\end{equation}

\begin{equation}
\ F_{B_y} = m_{BA}\alpha_{BA} + m_A\alpha_{BA}l + m_{BA}g + m_Ag
\end{equation}

\noindent En ce qui concerne la somme des moments d'inertie, 

\begin{equation}
\sum \overrightarrow{M_A} = I_A\alpha_A + I_{BA}\alpha_{BA}
\end{equation}

\begin{equation}
\ I_A\alpha_A + I_{BA}\alpha_{BA} = C_B - M_B - M_A
\end{equation}

\begin{equation}
\ C_B = \alpha_A + I_{BA}\alpha_{BA} + M_B + M_A
\end{equation}

\begin{equation}
\ C_B = (ml^2 + \frac{ml^2}{3})\alpha{BA}+ m_{BA}g\frac{l}{2}cos(\phi) + m_Aglcos(\phi)
\end{equation}

\subsection{Courbes Obtenue de Statique et de Dynamique}
Statique:
\newline
\noindent \includegraphics[width=\textwidth]{statique}
\newline
\noindent Dynamique:
\newline
\noindent \includegraphics[width=\textwidth]{dynamique}

\subsubsection{Analyse des courbes de Statique et Dynamique}
À première vue, on remarque que la courbe dynamique est plus élevée que la courbe statique. Cette observation est attendue, car afin de faire bouger un objet, il faut qu'il soit en équilibre (statique) et qu'on ajoute une force additionnelle. Il peut aussi être observé que la courbe dynamique semble représenter une simple translation positive sur l'axe des y de la courbe statique. Ce phénomène pourrait être expliqué par le fait qu'en statique, la somme des forces est égale à 0, mais qu'en dynamique, la somme des forces est non nulle. Cette différence entre la somme des forces en statique et dynamique semble représenter la valeur numérique de la translation de la courbe statique vers le courbe dynamique.


\end{document}